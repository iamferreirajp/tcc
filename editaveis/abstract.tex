\begin{resumo}[Abstract]
 \begin{otherlanguage*}{english}
   The present study presents the viability of the evaluation of the digitalized services on government. With the observed dificulty on literature about services qualification and the implication of this matter on the federal government, it has been verified the needs of the production and publication of this study. Therefore, were listed the standard guidelines for a real digital government, the service definition and the digitalized services, as well as the ways of evaluation and the usability definitions to software engeenering. A sistematic revision was initially conducted to raise theorectical referential and the SNOWBAL technique was used to this goal as well. With this research is expected, on the second stage, that is possible to obtain and prove by testing a model of evaluation of digitalized services in a government environmental.

   \vspace{\onelineskip}
 
   \noindent 
   \textbf{Key-words}: digitalization. service evaluation. government services.
 \end{otherlanguage*}
\end{resumo}

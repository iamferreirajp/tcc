\chapter[Serviços eletrônicos]{Serviços Eletrônicos}

\section{Definição}
\cite{santos2003} explica que a noção de serviços eletrônicos ou \textit{e-services} vem sendo reconhecida como fator determinante em comércios online ou \textit{e-commerce}, ao passo que \cite{rust2014} define que serviços eletrônicos podem ser explicados como sendo a expressão do serviço no ciberespaço.
\cite{rowley2006} traz definições sobre o serviço online que enriquece esse estudo. O serviço eletrônico pode ser definido como um serviço baseado na \textit{web} ou serviços interativos que são fornecidos pela internet. 

Como \cite{rowley2006} explica, vários autores conceituaram o serviço eletrônico como um serviço de informação, ou como \textit{self-service} de informações, onde as informações são fornecidas quando o usuário necessita. Na concepção do serviço eletrônico o consumidor tem uma interação diferenciada, não há contato cara a cara e toda experiência é realizada atravez da tecnologia pela qual o serviço é disponibilizado, portanto ao utilizar um serviço eletrônico, o usuário consume uma experiência e não um produto ou serviço físico. 

O usuário é parte fundamental no processo de aferição da qualidade de serviço digitizado, sua percepção, além de outros fatores, deve ser analisada para que se tenha um índice de qualidade correto.

\section{Qualidade de Serviços Eletrônicos}

\cite{buckley2003} traz uma compilação de três autores definindo a medição da qualidade de serviço, a definição de \cite{gefen2002} que sugere que ao avaliar a qualidade de serviços online as cinco dimensões do \textit{SERVQUAL} devem se tornar três: 
\begin{enumerate}
	\item Tangibilidade;
	\item Dimensão combinada entre responsividade, confiança e segurança;
	\item Empatia 
\end{enumerate}

Já \cite{chaffey2002} concluem que o \textit{SERVQUAL} por inteiro é aplicável e válido para medir a qualidade de serviços online, entretanto sugerem que a performance técnica do website, a clareza, a relevância das informações e acessibilidade são importantes nesta aferição. 

\cite{parasuraman2002} também analisam a qualidade do serviço eletrônico e define que há onze dimensões na qualidade de serviço eletrônico:
\begin{enumerate}
	\item Acesso;
	\item Facilidade de navegação;
	\item Eficiência;
	\item Personalização;
	\item Segurança/privacidade;
	\item Responsividade;
	\item Confiança;
	\item Conhecimento de preço;
	\item Estética do website;
	\item Confiança;
	\item Flexibilidade;
\end{enumerate}

Foi realizada então uma pesquisa com 540 usuários de internet e constatou-se que havia somente quatro dimensões principais:

\begin{enumerate}
	\item Eficiência;
	\item Completude;
	\item Confiabilidade;
	\item Privacidade;
\end{enumerate}

\section{Instrumentos de Aferição de Qualidade de Serviços Eletrônicos}

Muitos autores desenvolveram técnicas para aferir qualidade de serviços, como os já mostrados SERVQUAL e SERVPERF, e outros foram além para desenvolver uma maneira efetiva de medir qualidade de serviços eletrônicos. Dentre esses destaco 3 autores que desenvolveram instrumentos capazes de avaliar várias dimensões da qualidade inerente a um serviço eletrônico. Os autores trazem a visão da qualidade para a área do comércio virtual e com foco na percepção do consumidor.

\cite{lee2005} tem como objetivo desenvolver um modelo de pesquisa para examinar as relações entre as dimensões de qualidade dos serviços eletrônicos com qualidade de serviços normais, satisfação do cliente e intenções de compra. Os autores realizaram pesquisa com 297 consumidores de serviços eletrônicos pela internet e validaram o modelo com fatores confirmatórios explicados em seu trabalho. O instrumento desenvolvido pelos autores foi um questionário de afirmações demonstrado na tabela \ref{table:lee2005}, que traz várias dimensões de qualidade aplicados a um contexto de serviço de venda de livros na internet.

Já \cite{parasuraman2000}, responsáveis pelo modelo \textit{SERVQUAL}, também desenvolveram um modelo para avaliar serviços eletrônicos, esse modelo foi chamado de \textit{Eletronic Service Quality (E-S-QUAL)}. O questionário afirmativo dos autores tem 8 dimensões de qualidade, totalizando 33 afirmações que podem ser verificadas na tabela \ref{table:parasuraman2000}. É verificável a preocupação dos autores com muitas características não abordadas por \cite{lee2005}, como eficiência e compensação, porém algumas dimensões se equivalem e isso também acontece com os próximos autores.

\cite{zhang2005} tem como objetivo em seu trabalho estudar as relações entre os fatores que contribuem para a qualidade de serviços eletrônicos e a atitude dos consumidores para com esses serviços. Então desenvolveram outra ferramenta de aferição de qualidade, mas dessa vez o foco estava na perspectiva do consumidor dessa qualidade, o questionário afirmativo destes autores conta com 20 afirmações distribuídas em 5 dimensões verificáveis na Tabela \ref{zhang2005}. Pode-se afirmar que o viés de pesquisa desses autores enalteceram a percepção do consumidor em detrimento das características puramente técnicas das plataformas e dos serviços, levando assim a uma qualidade percebida medida principalmente pela satisfação dos usuários destes serviços.
\chapter[Introdução]{Introdução}
% \addcontentsline{toc}{chapter}{Introdução}
\section{Contextualização}
A integração estável de novas tecnologias (computação em nuvem, mídias sociais, tecnologias móveis) no dia-a-dia das pessoas, negócios e governos está ajudando a abrir portas e dar origem a novas formas de engajamento do público  e relações que transcendem as esferas pública, privada e social.\cite{oecd2014}

A difusão e adoção de tecnologias está mudando as expectativas na habilidade dos governos em entregar valor ao público. Os governos não podem mais separar eficiência dos outros objetivos político-sociais e gerenciamento de tecnologias digitais. A crise econômica e financeira mostra que a entrega melhorada de serviços e eficiência interna no setor público anda de mãos dadas com o crescimento econômico, igualdade social e bons objetivos de governança como uma melhor transparência, integridade e engajamento do cidadão.\cite{oecd2014}

A importância dessas novas tecnologias deram origem ao termo “digitizar”. Para \cite{steven2015}, digitização é o processo de mudança de dados para um formato digital que seja facilmente lido e processado por um computador, mas o processo de digitizar vai além de somente dados, os serviços também são transformados em serviços digitais.

Há poucos governos enfrentando o mesmo tipo de pressão competitiva que obriga as empresas e mesmo as organizações do setor social a digitizar. Pois há um risco que o governo seja identificado como um desafiante digital. Mas isso não deve ser uma desculpa para a complacência governamental. As dinâmicas da globalização exigem que países e municipalidades devem ter investimentos, mão-de-obra e conhecimento, que são recursos dos quais tecnologias digitais podem ser imãs. \cite{mckinsey2016}

Estes recursos são utilizados para fornecimento de serviços e esses necessitam de qualidade no seu fornecimento. Segundo \cite{cronintaylor1992} o interesse na medição de qualidade de serviços é alto e a entrega de níveis altos na qualidade do serviço é a estratégia que vem melhor posicionando os fornecedores de serviço no mercado. 

Portanto, este trabalho visa estudar a qualidade dos serviços digitizados na esfera governamental para que, conhecendo suas características, seja possível a obtenção de um modelo de avaliação baseado em metodologias pré-existentes. 


\section{Questões de pesquisa}

Dado o contexto, justificativa do presente tema e com o propósito de guiar a linha de execução do trabalho, foram elaboradas as seguintes questões de pesquisa:
\begin{enumerate}
	\item De que maneira a avaliação de serviços digitizados auxilia na aferição da qualidade da digitização dos processos do serviço?

	\item Como obter um modelo de avaliação de serviços governamentais digitizados?

	\item Quais os métodos de avaliação de serviço que permitem a aferição da qualidade de serviços digitizados?
\end{enumerate}

\section{Objetivos}
\subsection{Objetivo geral}
	O presente trabalho tem como objetivo geral avaliar a qualidade dos serviços governamentais digitizados, tendo como benefício a obtenção de um modelo de avaliação de serviços governamentais digitizados no governo federal brasileiro.

\subsection{Objetivos específicos}

\subsubsection{Referencial teórico}
\begin{itemize}
\item 	Conceituar - Governo digital
\item	Conceituar - Serviço
\item	Conceituar - Avaliação de serviços
\item	Conceituar - Digitização de serviços
\item	Descrever - Avaliação de digitização
\item	Diferenciar - Critérios de qualidade
\item 	Propor - Modelo de avaliação de serviços digitizados governamentais
\end{itemize}

\subsubsection{Metodologia}
	Estratégia de contrução da fundamentação teórica:
		Revisão sistemática sobre digitização governamental.

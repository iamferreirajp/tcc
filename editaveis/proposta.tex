\chapter[Proposta de modelo]{Proposta de modelo}
Neste capítulo serão descritas as características que influenciarão na obtenção do modelo de avaliação que é o foco deste trabalho. Serão descritas as fase para a obtenção deste modelo e uma prévia do que o modelo se tornará.

\section{Avaliação de qualidade de serviços digitizados governamentais}
No trabalho de \cite{cronintaylor1992} são definidas questões pertinentes para a avaliação de um serviço tradicional. Para a avaliação de um serviço digitizado é necessário que as dimensões levantadas pelos autores sejam postas à prova para que possam mensurar a qualidade esses seviços, visto que suas características diferem das apresentadas pelos serviços tradicionais.

A proposta é que um modelo baseado no SERVPERF com dimensões levantadas pela avaliação de IHC nos testes de usabilidade seja construído para a realização da segunda parte do trabalho. Cada campo de análise do SERVPERF será analisado para verificar sua relevância na avaliação de qualidade dos serviços governamentais digitizados.

No capítulo de materiais e métodos serão descritas as metodologias de construção e validação do modelo proposto.

\subsubsection{Estatísticas relacionadas}

A análise quantitativa e qualitativa sdo modelo será realizada tendo como base o SERVPERF, o SERVPERF balanceado, o SERVQUAL e o SERVQUAL balanceados, os quais darão insumos e comparações para a validação do modelo proposto neste trabalho. Apesar de ainda estar em estado de análise já é possível levantar que as relevâncias de cada área serão levantadas para cada usuário e atuarão como multiplicadores das pontuações de cada questão dentro da avaliação, culminando em uma pesquisa por meio de questionário balanceada pela percepção do usuário do serviço prestado.
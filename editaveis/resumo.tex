\begin{resumo}

O presente trabalho apresenta um estudo de viabilidade da avaliação de serviços digitizados governamentais. A partir da dificuldade observada na literatura sobre qualificação de serviços e a implicação disso no governo federal, verificou-se a necessidade da produção e publicação deste estudo. Para tanto, elencou-se os padrões para um governo digital, as definições de serviços e serviços digitizados, bem como suas formas de avaliação e as definições de usabilidade para engenharia de software. Uma revisão sistemática foi conduzida inicialmente para levantamento de referencial teórico e também foi utilizada a técnica \textit{SNOWBALL} para este fim. Com esta pesquisa espera-se, na segunda etapa, que seja possível a obtenção e teste de um modelo de avaliação de serviços digitizados em um ambiente governamental. 

 \vspace{\onelineskip}
    
 \noindent
 \textbf{Palavras-chave}: digitizacão. avaliação de serviços. serviços governamentais.
\end{resumo}

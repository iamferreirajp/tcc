\chapter[Metodologia]{Metodologia}
Neste capítulo serão apresentados os métodos e processos de avaliação, os quais passarão por validação e deverão ser aplicados em um contexto real de avaliação, respondendo assim a questão de pesquisa: Como obter um modelo de avaliação de serviços governamentais digitizados?

\section{Referencial teórico}

Para o referencial teórico foram definidas as seguintes atividades para serem desenvolvidas:

\begin{itemize}
\item 	Conceituar - Governo digital
\item	Conceituar - Serviço
\item	Conceituar - Avaliação de serviços
\item	Conceituar - Digitização de serviços
\item	Descrever - Avaliação de digitização
\item	Diferenciar - Critérios de qualidade
\item 	Propor - Modelo de avaliação de serviços digitizados governamentais
\end{itemize}

\section{Materiais e métodos}
	\subsection{Validação do modelo proposto}
	O modelo proposto por esse trabalho será posto à prova primeiramente durante uma avaliação de especialistas. Os professores da faculdade UNB do Gama que têm a avaliação de serviços, IHC (Interação humano-computador) e modelos de pesquisa como seus campos de pesquisa serão convidados a avaliar o modelo construído, de modo a validar ou propor modificações em seu método de construção.

	Logo após a validação feita pelos especialistas, o modelo deve ser colocado à prova em um teste-piloto, onde será avaliado um serviço digitizado amplamente utilizado por alunos da faculdade FGA, como por exemplo o matrícula WEB. Os serviços digitizados fornecidos pela UNB têm a assinatura governamental por se tratarem de uma instituição federal, portanto, são serviços passíveis de serem avaliados pelo modelo proposto nesse trabalho.

	Os resultados serão apresentados em contraste com outros métodos de avaliação para assegurar sua validade e serão apresentados novamente aos especialistas. Desta forma, o modelo será consolidado e estará pronto para a avaliação de um serviço digitizado governamental real.

	\subsection{Planejamento inicial da avaliação de serviços}
	Espera-se que na fase após a validação do teste-piloto, o teste real possa ser realizado. Diante da proposta do Ministério do Planejamento da digitização dos serviços dos ministérios, é possível que o modelo seja posto à prova durante a digitização dos serviços acompanhados pelo MP.

	\subsection{Descrição das ferramentas utilizadas}
	A avaliação poderá ser feita de duas maneiras:
	\begin{itemize}
		\item Questionário de avaliação: um questionário de avaliação adaptado do SERVPERF com dimensões de usabilidade será respondido pelos usuários dos serviços que estarão sendo avaliados. Com esta forma de avaliação se terá um alcance maior, porém menor controle sobre as condições do teste realizado pelos usuários;
		
		\item Avaliação de uso: o questionário será preenchido por um avaliador analisando o usuário durante o uso do sistema. Esta forma de avaliação terá menor alcance, mas garantirá padronização da condução dos testes.
	\end{itemize}

	O modelo de avaliação terá de ser personalizado para cada uso, visto que cada segmento de serviço digitizado pode tratar de temas específicos e o contexto é um item importante a ser avaliado dentro do modelo.
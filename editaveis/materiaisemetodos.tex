\chapter[Metodologia]{Metodologia}
Neste capítulo serão apresentados os métodos e processos de avaliação, respondendo assim a questão de pesquisa: Como obter um modelo de avaliação de serviços governamentais digitizados?

\section{Materiais e métodos}
	\subsection{Contexto de Avaliação}

	Para realizar a avaliação real de um serviço digitizado governamental foi escolhido como foco do estudo o sistema MatrículaWeb, por ser uma ferramenta utilizada por todos os estudantes da Universidade de Brasília e pelo contexto governamental onde se encontra. Portanto todas as atividades de validação do modelo de avaliação serão realizadas utilizando esta plataforma de serviços como caso de estudo real.

	\subsection{Obtenção do modelo de avaliação}

	Como apresentado no capítulo anterior, existem muitas ferramentas de avaliação de serviços eletrônicos, porém todas foram geradas a partir de uma linha de pensamento específico, analisam principalmente os comércios eletrônicos, o que na época era o modelo de serviço eletrônico disponível. Nesta pesquisa o foco é a aferição da qualidade de serviços governamentais digitizados, portanto algumas dimensões de ferramentas voltadas ao comércio online não são aptas a descrever esse cenário.

	A partir dos três modelos apresentados no capítulo anterior realizou-se um estudo para identificar semelhanças e discrepâncias entre as dimensões e questões dos modelos. Foram realizadas as seguintes etapas:

	\begin{itemize}
		\item Identificar os temas de todas as questões;
		\item Remover as questões que não são aptas a avaliar o contexto, como por exemplo questões de cartão de crédito;
		\item Agrupar as questões com os mesmos temas e verificar semelhanças;
		\item Formular novas questões para os temas repetidos, de forma que haja somente 1 questão por tema;
		\item Realizar análise semântica das questões com voluntários e validações de entendimento.
	\end{itemize}

	\subsubsection{Identificação dos temas}

	Utilizando as três ferramentas já apresentadas foram identificados os temas de cada questão conforme as Tabelas \ref{table:analiselee2005}, \ref{table:analiseparasuraman2000} e \ref{table:analisezhang2005}. A identificação desses temas é importante para agruparmos as questões, que foram escritas de formas diferentes, em conjuntos em concordância com seus respectivos temas, dessa forma podemos analisar todos os questionários uniformemente.


	\subsubsection{Remoção das questões inaptas a avaliar o contexto}

	Com os temas identificados foi realizada a análise das questões que não condizem com o contexto governamental que é o foco do estudo. A partir dessa análise foram removidas dos questionários as seguintes questões:
	
	Do questionário de \cite{lee2005}:
	\begin{itemize}
		\item {The online bookstore delivers on its undertaking to do certain things by a certain time.	-	Comprometimento da empresa com alguma garantia.} 
	\end{itemize}
	
	Do questionário de \cite{parasuraman2000}:
	\begin{itemize}
		\item {FUL1 It delivers orders when promised.	-	Capacidade de entregar o que promete} 
		\item {FUL3 It quickly delivers what I order.	-	Rapidamente entrega o pedido}
		\item {FUL4 It sends out the items ordered.	-	Envia os itens requisitados}
		\item {FUL5 It has in stock the items the company claims to have.	-	Tem em estoque o que promete}
		\item {FUL7 It makes accurate promises about delivery of products.	-	Promessas realistas sobre a entrega do produto}
		\item {PRI3 This site protects information about my credit card.	-	Segurança das informações do cartão de crédito}
		\item {RES1 It provides me with convenient options for returning items.	-	Fornecimento de maneiras de devolução}
		\item {RES2 This site handles product returns well.	-	Boa resposta com devoluções}
		\item {RES3 This site offers a meaningful guarantee.	-	Garantia com significado real}
		\item {COM2 It compensates me when what I ordered doesn’t arrive on time.	-	Compensação de entrega atrasada}
		\item {COM3 It picks up items I want to return from my home or business.	-	Capacidade de lidar com devoluções desde a casa do consumidor}
	\end{itemize}
	
	Do questionário de \cite{zhang2005}:
	\begin{itemize}
		\item {The information about the products for your needs/interests is sufficient  for me to make a purchase decision.	-	Informações suficientes sobre os interesses dos usuários para decisão de comprar} 
		\item {I am happy with the vendor guaranteed policy.	-	Felicidade com a política de garantia}
		\item {The vendor gives prompt service to customers.	-	A prontidão do vendedor }
		\item {I worry about credit card information being stolen on internet.	-	Preocupação sobre roubo de informações do cartão}
	\end{itemize}

	Portanto, foram retiradas ao todo 16 questões das 62 dos 3 questionários.

	\subsubsection{Agrupamento de Questões}

	Com as questões fora de contexto removidas, as que restaram foram agrupadas por seus temas, como pode ser observado na Tabela \ref{table:questoesagrupadasportemas}. Essa junção e observação auxilia na verificação e formulação de novas questões.

	\subsubsection{Formulação das novas questões}

	Neste passo foram formuladas novas questões baseadas nos temas e questões de cada autor, de forma que pudessem refletir o contexto analisado. Durante a formulação ainda foram encontradas questões e temas que não condiziam com o cenário e contexto aplicado e também foram removidas. Foram formuladas 29 questões baseadas nos temas e questões anteriores, mas aplicadas ao contexto de estudo. O resultado pode ser conferido na Tabela \ref{table:novasquestoes}.

	\subsubsection{Análise semântica das questões formuladas}

	As questões obtidas no passo anterior ainda precisavam ser validadas com pessoas reais, para que pudessem estar livres de outras interpretações, portanto foi realizado um estudo com grupo focal de 5 alunos da UnB que leram as questões e opinaram sobre melhorias nos textos para que ficassem mais explícitos e fáceis de entender. O resultado dessa análise é um questionário reescrito pronto para ser aplicado aos alunos da UnB, que pode ser conferido na Tabela \ref{table:questoesrevisadas}.

	\subsection{Ferramenta para Aplicação do Questionário}

	Para a coleta de informações foi escolhida a ferramenta Google Forms, por ser de fácil manuseio e acesso por todos os estudantes da UnB. O documento final do questionário pode ser encontrado no Apêndice \ref{questionariogoogle}.

	Foi escolhida o tipo de resposta de 1 a 5, onde 1 significa discordo totalmente e 5 concordo totalmente, as questões foram formuladas como afirmações, para que o respondente possa manifestar sua concordância com a afirmação, desta forma pode-se analisar o sentimento do respondente em relação a cada afirmação. Este tipo de questionário também foi utilizado por \cite{parasuraman2000} na aplicação do \textit{E-S-QUAL}.

\section{Análise de Dados}

	O questionário de avaliação recebeu 74 respostas de alunos da UnB dos 4 campi, essas informações foram analisadas para verificar inconsistências e constatou-se que não houve discrepâncias de respostas erradas, faltantes ou alguma anomalia nos resultados.

	Para \cite{almeida2007} a análise fatorial é frequentemente utilizando para identificar estrutura dos dados. Para realizar a análise fatorial nesta pesquisa foram realizadas as análises descritiva e exploratória de dados com o software SPSS \textit{(Statistical Package for Social Sciences)} versão 22, com a ajuda do instrumento foi realizada a análise de componentes principais (PC) com objetivo de verificar a fatorabilidade do instrumento de avaliação e estabelecer a quantidade de fatores possíveis a serem testados.

\subsection{Fatorabillidade da matriz de correlação}
De acordo com \cite{kaiser1974} os índices fatoriais podem ser avaliados de acordo com a seguinte tabela:

% Please add the following required packages to your document preamble:
% \usepackage{graphicx}
\begin{table}[H]
\centering
\caption{Avaliação dos índices fatoriais. Fonte: \cite{kaiser1974}}
\label{table:indicesfatoriais}
% \resizebox{\textwidth}{!}{%
\begin{tabular}{|l|l|}
\hline
acima de 0,90  & Maravilhoso \\ \hline
acima de 0,80  & Meritório   \\ \hline
acima de 0,70  & Mediano     \\ \hline
acima de 0,60  & Medíocre    \\ \hline
acima de 0,50  & Deplorável  \\ \hline
abaixo de 0,50 & Inaceitável \\ \hline
\end{tabular}%
% }
\end{table}


Índices de KMO e teste de esfericidade de Bartlett

\subsection{Quantidade de fatores possíveis}
Variância explicada e \textit{engenvalues}

\subsection{Análise fatorial}
\textit{Principal Axis Factoring} (PAF) e rotação Oblíqua \textit{(Direct Oblimin)}

Para todos os itens da análise fatorial existe uma carga fatorial relacionada, os itens com carga fatorial abaixo de 0,40 foram descartados, pois, de acordo com a interpretação desses valores, esses itens não possuem importância na avaliação do contexto. Para a consideração do pertencimento de uma variável a um determinado fator o critério é definido pelo pesquisador, mas de acordo com \cite{andreoli1994} muitos trabalhos utilizam a carga fatorial acima de 0,40 ou 0,50.
\begin{anexosenv}

\partanexos

\chapter{Primeiro Anexo}
\pagebreak
% Please add the following required packages to your document preamble:
% \usepackage[table,xcdraw]{xcolor}
% If you use beamer only pass "xcolor=table" option, i.e. \documentclass[xcolor=table]{beamer}
\begin{table}[]
\centering
\caption{Proposta de questionário para análise de qualidade de serviço digitizado. Fonte: o autor.}
\label{table:propostaquestionario}
\resizebox{1.1\textwidth}{!}{%
\begin{tabular}{ccc}
\hline
\multicolumn{1}{|c|}{} & \multicolumn{1}{c|}{\cellcolor[HTML]{9B9B9B}SERVPERF} & \multicolumn{1}{c|}{\cellcolor[HTML]{9B9B9B}Usabilidade} \\ \hline
\rowcolor[HTML]{C0C0C0} 
\multicolumn{1}{|c|}{\cellcolor[HTML]{C0C0C0}Dimensão} & \multicolumn{1}{c|}{\cellcolor[HTML]{C0C0C0}Tangibilidade} & \multicolumn{1}{c|}{\cellcolor[HTML]{C0C0C0}Inteligibilidade/Apreensibilidade} \\ \hline
\multicolumn{1}{|c|}{1} & \multicolumn{1}{c|}{Estado dos equipamentos.} & \multicolumn{1}{c|}{O conteúdo textual está claro e consistente} \\ \hline
\multicolumn{1}{|c|}{2} & \multicolumn{1}{c|}{\begin{tabular}[c]{@{}c@{}}As instalações físicas \\ são visualmente agradáveis.\end{tabular}} & \multicolumn{1}{c|}{\begin{tabular}[c]{@{}c@{}}Este sistema tem uma apresentação gráfica \\ agradável e legível?\end{tabular}} \\ \hline
\multicolumn{1}{|c|}{3} & \multicolumn{1}{c|}{Funcionários bem vestidos.} & \multicolumn{1}{c|}{\begin{tabular}[c]{@{}c@{}}Os recursos de navegação (menus, ícones, links e botões) \\ estão todos claros e fáceis de achar.\end{tabular}} \\ \hline
\multicolumn{1}{|c|}{4} & \multicolumn{1}{c|}{\begin{tabular}[c]{@{}c@{}}Aparência das instalações \\ são consistentes com o \\ tipo de serviço prestado.\end{tabular}} & \multicolumn{1}{c|}{Logo que entro no site já sei o que esta empresa faz.} \\ \hline
\rowcolor[HTML]{C0C0C0} 
\multicolumn{1}{|c|}{\cellcolor[HTML]{C0C0C0}Dimensão} & \multicolumn{1}{c|}{\cellcolor[HTML]{C0C0C0}Confiabilidade} & \multicolumn{1}{c|}{\cellcolor[HTML]{C0C0C0}Conformidade} \\ \hline
\multicolumn{1}{|c|}{5} & \multicolumn{1}{c|}{\begin{tabular}[c]{@{}c@{}}O fornecedor cumpre o \\ prazo de resposta prometido.\end{tabular}} & \multicolumn{1}{c|}{A apresentação das páginas com o serviço é rápida} \\ \hline
\multicolumn{1}{|c|}{6} & \multicolumn{1}{c|}{O fornecedor é simpático e tranquilizador quando o consumidor tem problemas.} & \multicolumn{1}{c|}{\begin{tabular}[c]{@{}c@{}}Possuo todas as informações necessárias \\ quando estou utilizando o serviço.\end{tabular}} \\ \hline
\multicolumn{1}{|c|}{7} & \multicolumn{1}{c|}{O fornecedor é confiável} & \multicolumn{1}{c|}{\begin{tabular}[c]{@{}c@{}}As informações fornecidas pelo serviço \\ são confiáveis?\end{tabular}} \\ \hline
\multicolumn{1}{|c|}{8} & \multicolumn{1}{c|}{O fornecedor fornece o serviço quando prometido} & \multicolumn{1}{c|}{\begin{tabular}[c]{@{}c@{}}Todos os serviços e produtos estão classificados \\ claramente e de acordo com o seu segmento.\end{tabular}} \\ \hline
\multicolumn{1}{|c|}{9} & \multicolumn{1}{c|}{O fornecedor mantém registros corretos} & \multicolumn{1}{c|}{\begin{tabular}[c]{@{}c@{}}A lista com produtos é clara e possui \\ todas as informações necessárias.\end{tabular}} \\ \hline
\rowcolor[HTML]{C0C0C0} 
\multicolumn{1}{|c|}{\cellcolor[HTML]{C0C0C0}Dimensão} & \multicolumn{1}{c|}{\cellcolor[HTML]{C0C0C0}Presteza} & \multicolumn{1}{c|}{\cellcolor[HTML]{C0C0C0}Operacionalidade} \\ \hline
\multicolumn{1}{|c|}{10} & \multicolumn{1}{c|}{\begin{tabular}[c]{@{}c@{}}Não é esperado que o fornecedor diga ao consumidor exatamente \\ quando o serviço será realizado. (Negativa)\end{tabular}} & \multicolumn{1}{c|}{\begin{tabular}[c]{@{}c@{}}Não se espera uma informação de \\ quanto tempo o serviço levará.\end{tabular}} \\ \hline
\multicolumn{1}{|c|}{11} & \multicolumn{1}{c|}{\begin{tabular}[c]{@{}c@{}}Não é razoável esperar os serviços rápidos \\ dos funcionários. (Negativa)\end{tabular}} & \multicolumn{1}{c|}{\begin{tabular}[c]{@{}c@{}}Não é esperado um fornecimento \\ de serviço rápido.\end{tabular}} \\ \hline
\multicolumn{1}{|c|}{12} & \multicolumn{1}{c|}{\begin{tabular}[c]{@{}c@{}}Funcionários nem sempre se dispõem a ajudar \\ os consumidores. (Negativa)\end{tabular}} & \multicolumn{1}{c|}{\begin{tabular}[c]{@{}c@{}}Não existe página de ajuda ou ela está \\ em um lugar de difícil acesso.\end{tabular}} \\ \hline
\multicolumn{1}{|c|}{13} & \multicolumn{1}{c|}{\begin{tabular}[c]{@{}c@{}}Está tudo bem em estar muito ocupado para responder \\ prontamente à uma requisição do consumidor. (Negativa)\end{tabular}} & \multicolumn{1}{c|}{\begin{tabular}[c]{@{}c@{}}O sistema não funcionou ou \\ houveram travamentos.\end{tabular}} \\ \hline
\rowcolor[HTML]{C0C0C0} 
\multicolumn{1}{|c|}{\cellcolor[HTML]{C0C0C0}Dimensão} & \multicolumn{1}{c|}{\cellcolor[HTML]{C0C0C0}Segurança} & \multicolumn{1}{c|}{\cellcolor[HTML]{C0C0C0}Conformidade} \\ \hline
\multicolumn{1}{|c|}{14} & \multicolumn{1}{c|}{Funcionários devem ser confiáveis} & \multicolumn{1}{c|}{Eu confio no sistema.} \\ \hline
\multicolumn{1}{|c|}{15} & \multicolumn{1}{c|}{\begin{tabular}[c]{@{}c@{}}Os consumidores devem ser sentir seguros quando \\ estiverem lidando com os funcionários.\end{tabular}} & \multicolumn{1}{c|}{Eu me sinto seguro nesse sistema.} \\ \hline
\multicolumn{1}{|c|}{16} & \multicolumn{1}{c|}{Funcionários devem ser corteses} & \multicolumn{1}{c|}{-} \\ \hline
\multicolumn{1}{|c|}{17} & \multicolumn{1}{c|}{\begin{tabular}[c]{@{}c@{}}Funcionários devem ter suporte adequando de suas organizações \\ para fazer bem o seu trabalho.\end{tabular}} & \multicolumn{1}{c|}{\begin{tabular}[c]{@{}c@{}}O sistema pediu muitas informações, \\ que talvez não seriam utilizadas?\end{tabular}} \\ \hline
\rowcolor[HTML]{C0C0C0} 
\multicolumn{1}{|c|}{\cellcolor[HTML]{C0C0C0}Dimensão} & \multicolumn{1}{c|}{\cellcolor[HTML]{C0C0C0}Empatia} & \multicolumn{1}{c|}{\cellcolor[HTML]{C0C0C0}Atratividade} \\ \hline
\multicolumn{1}{|c|}{18} & \multicolumn{1}{c|}{\begin{tabular}[c]{@{}c@{}}As organizações não devem dar atenção \\ individualizada ao consumidor (Negativa)\end{tabular}} & \multicolumn{1}{c|}{-} \\ \hline
\multicolumn{1}{|c|}{19} & \multicolumn{1}{c|}{\begin{tabular}[c]{@{}c@{}}Não se espera dos funcionários que eles deem atenção \\ individual a cada consumidor. (Negativa)\end{tabular}} & \multicolumn{1}{c|}{-} \\ \hline
\multicolumn{1}{|c|}{20} & \multicolumn{1}{c|}{\begin{tabular}[c]{@{}c@{}}Está fora da realidade esperar que os funcionários \\ entendam as necessidades dos consumidores. (Negativa)\end{tabular}} & \multicolumn{1}{c|}{Não está claro se o sistema é melhorado baseado nas avaliações dos consumidores.} \\ \hline
\multicolumn{1}{|c|}{21} & \multicolumn{1}{c|}{\begin{tabular}[c]{@{}c@{}}Está fora de questão esperar que os funcionários se preocupem \\ com os interesses dos consumidores. (Negativa)\end{tabular}} & \multicolumn{1}{c|}{O sistema não tem páginas ou meios de comunicação para relatar reclamações ou melhorias.} \\ \hline
\multicolumn{1}{|c|}{22} & \multicolumn{1}{c|}{\begin{tabular}[c]{@{}c@{}}As organizações não necessariamente devem operar \\ nas horas convenientes a todos os consumidores. (Negativa)\end{tabular}} & \multicolumn{1}{c|}{O sistema estava fora do ar.} \\ \hline
\rowcolor[HTML]{C0C0C0} 
\multicolumn{1}{|c|}{\cellcolor[HTML]{C0C0C0}} & \multicolumn{2}{c|}{\cellcolor[HTML]{C0C0C0}Questões adicionais} \\ \hline
\multicolumn{1}{|c|}{23} & \multicolumn{1}{c|}{Você utilizou o sistema nos últimos 6 meses?} & \multicolumn{1}{c|}{-} \\ \hline
\multicolumn{1}{|c|}{24} & \multicolumn{1}{c|}{Você voltaria utilizar esse sistema?} & \multicolumn{1}{c|}{-} \\ \hline
\multicolumn{1}{|c|}{25} & \multicolumn{1}{c|}{Qual a probabilidade de você recomendar o sistema para alguém?} & \multicolumn{1}{c|}{-} \\ \hline
\multicolumn{1}{l}{} & \multicolumn{1}{l}{} & \multicolumn{1}{l}{}
\end{tabular}
}
\end{table}

\chapter{Segundo Anexo}
\pagebreak
% Please add the following required packages to your document preamble:
% \usepackage[table,xcdraw]{xcolor}
% If you use beamer only pass "xcolor=table" option, i.e. \documentclass[xcolor=table]{beamer}
\begin{table}[]
\centering
\caption{Segunda Proposta de avaliação de serviços. Fonte: ESQUAL adaptado pelo autor}
\label{table:propostaquestionario2}
\begin{tabular}{|l|l|l|}
\hline
\multicolumn{3}{|c|}{\cellcolor[HTML]{9B9B9B}{\color[HTML]{333333} Questionário de avaliação de serviço}} \\ \hline
 & \multicolumn{1}{c|}{\cellcolor[HTML]{C0C0C0}{\color[HTML]{333333} Característica}} & \multicolumn{1}{c|}{\cellcolor[HTML]{C0C0C0}{\color[HTML]{333333} Avaliação}} \\ \hline
\multicolumn{3}{|c|}{\cellcolor[HTML]{9B9B9B}Eficiência} \\ \hline
EFF1 & É fácil achar o que eu quero nesse serviço. & \begin{tabular}[c]{@{}l@{}}Discordo Totalmente\\ Discordo Levemente\\ Neutro\\ Concordo Parcialmente\\ Concordo Totalmente\end{tabular} \\ \hline
EFF2 & É fácil ir à qualquer lugar no site. &  \\ \hline
EFF3 & \begin{tabular}[c]{@{}l@{}}O serviço me ajudou a completar o meu objetivo\\ rapidamente.\end{tabular} &  \\ \hline
EFF4 & \begin{tabular}[c]{@{}l@{}}As informações fornecidas pelo serviço são \\ bem organizadas.\end{tabular} &  \\ \hline
EFF5 & As páginas são carregadas rapidamente. &  \\ \hline
EFF6 & O serviço é simples de ser usado. &  \\ \hline
EFF7 & O serviço permite o meu avanço rapidamente. &  \\ \hline
EFF8 & O serviço é bem organizado. &  \\ \hline
\end{tabular}
\end{table}


\chapter{Terceiro Anexo}
\pagebreak
% Please add the following required packages to your document preamble:
% \usepackage{multirow}
% \usepackage[table,xcdraw]{xcolor}
% If you use beamer only pass "xcolor=table" option, i.e. \documentclass[xcolor=table]{beamer}
\begin{table}[]
\caption{Instrumento de aferição de qualidade de livraria online. Fonte: \cite{lee2005}}
\label{table:lee2005}
\resizebox{1.1\textwidth}{!}{%
\begin{tabular}{|l|l|l|}
\hline
\multicolumn{3}{|c|}{\cellcolor[HTML]{9B9B9B}{\color[HTML]{333333} Consumer perceptions of E-Service quality in online shopping (Gwo-Guang Lee)}} \\ \hline
\multicolumn{3}{|c|}{\cellcolor[HTML]{C0C0C0}Web Site Design} \\ \hline
\multicolumn{3}{|l|}{The online bookstore is visually appealing} \\ \hline
\multicolumn{3}{|l|}{The user interface of the online bookstore has a well-organized appearance} \\ \hline
\multicolumn{3}{|l|}{It is quick and easy to complete a transaction at the online bookstore} \\ \hline
\multicolumn{3}{|c|}{\cellcolor[HTML]{C0C0C0}} \\
\multicolumn{3}{|c|}{\multirow{-2}{*}{\cellcolor[HTML]{C0C0C0}Reliability}} \\ \hline
\multicolumn{3}{|l|}{The online bookstore delivers on its undertaking to do certain things by a certain time} \\ \hline
\multicolumn{3}{|l|}{The online bookstore shows a sincere interest in solving customer problems} \\ \hline
\multicolumn{3}{|l|}{Transactions with the online bookstore are error-free} \\ \hline
\multicolumn{3}{|l|}{The online bookstore has adequate security} \\ \hline
\multicolumn{3}{|c|}{\cellcolor[HTML]{C0C0C0}Responsiveness} \\ \hline
\multicolumn{3}{|l|}{I think the online bookstore gives prompt service} \\ \hline
\multicolumn{3}{|l|}{I believe the online bookstore is always willing to help customers} \\ \hline
\multicolumn{3}{|l|}{I believe the online bookstore is never too busy to respond to customer requests} \\ \hline
\multicolumn{3}{|c|}{\cellcolor[HTML]{C0C0C0}Trust} \\ \hline
\multicolumn{3}{|l|}{I believe the online bookstore is trustworthy} \\ \hline
\multicolumn{3}{|l|}{The online bookstore instills confidence in customers} \\ \hline
\multicolumn{3}{|c|}{\cellcolor[HTML]{C0C0C0}} \\
\multicolumn{3}{|c|}{\multirow{-2}{*}{\cellcolor[HTML]{C0C0C0}Personalization}} \\ \hline
\multicolumn{3}{|l|}{The online bookstore provides the targeting e-mail to customers} \\ \hline
\multicolumn{3}{|l|}{The online bookstore provides the recommendation of books by customers’ preferences} \\ \hline
\multicolumn{3}{|l|}{The online bookstore provides customers free personal homepage} \\ \hline
\multicolumn{3}{|c|}{\cellcolor[HTML]{C0C0C0}Overall service quality} \\ \hline
\multicolumn{3}{|l|}{My overall opinion of the services provided by online bookstore is very good} \\ \hline
\multicolumn{3}{|c|}{\cellcolor[HTML]{C0C0C0}Customer satisfaction} \\ \hline
\multicolumn{3}{|l|}{Overall, I am satisfied with online bookstore online experience} \\ \hline
\multicolumn{3}{|c|}{\cellcolor[HTML]{C0C0C0}Purchase intentions} \\ \hline
\multicolumn{3}{|l|}{If I purchase books in the next 30 days, I will use the online bookstore} \\ \hline
\multicolumn{3}{|l|}{I strongly recommend that others use the online bookstore} \\ \hline
\end{tabular}
}
\end{table}

\chapter{Quarto Anexo}
\pagebreak
% Please add the following required packages to your document preamble:
% \usepackage{graphicx}
% \usepackage[table,xcdraw]{xcolor}
% If you use beamer only pass "xcolor=table" option, i.e. \documentclass[xcolor=table]{beamer}
% \usepackage[normalem]{ulem}
% \useunder{\uline}{\ul}{}
\begin{table}[]
\caption{Instrumento de aferição de qualidade para serviços eletrônicos. Fonte: \cite{parasuraman2000}}
\label{table:parasuraman2000}
\resizebox{1.1\textwidth}{!}{%
\begin{tabular}{|l|}
\hline
\rowcolor[HTML]{9B9B9B} 
E-S-QUAL a Multiple-Item Scale for Assessing Eletronic Service Quality (Parasuraman) \\ \hline
\rowcolor[HTML]{C0C0C0} 
\multicolumn{1}{|c|}{\cellcolor[HTML]{C0C0C0}Efficiency} \\ \hline
EFF1 This site makes it easy to find what I need. \\ \hline
EFF2 It makes it easy to get anywhere on the site. \\ \hline
EFF3 It enables me to complete a transaction quickly. \\ \hline
EFF4 Information at this site is well organized \\ \hline
EFF5 It loads its pages fast. \\ \hline
EFF6 This site is simple to use. \\ \hline
EFF7 This site enables me to get on to it quickly. \\ \hline
EFF8 This site is well organized. \\ \hline
\rowcolor[HTML]{C0C0C0} 
\multicolumn{1}{|c|}{\cellcolor[HTML]{C0C0C0}System Availability} \\ \hline
SYS1 This site is always available for business. \\ \hline
SYS2 This site launches and runs right away. \\ \hline
SYS3 This site does not crash. \\ \hline
SYS4 Pages at this site do not freeze after I enter my order information. \\ \hline
\rowcolor[HTML]{C0C0C0} 
\multicolumn{1}{|c|}{\cellcolor[HTML]{C0C0C0}Fulfillment} \\ \hline
FUL1 It delivers orders when promised. \\ \hline
FUL2 This site makes items available for delivery within a suitable time frame. \\ \hline
FUL3 It quickly delivers what I order. \\ \hline
FUL4 It sends out the items ordered. \\ \hline
FUL5 It has in stock the items the company claims to have. \\ \hline
FUL6 It is truthful about its offerings \\ \hline
FUL7 It makes accurate promises about delivery of products. \\ \hline
\rowcolor[HTML]{C0C0C0} 
\multicolumn{1}{|c|}{\cellcolor[HTML]{C0C0C0}Privacy} \\ \hline
PRI1 It protects information about my Web-shopping behavior. \\ \hline
PRI2 It does not share my personal information with other sites. \\ \hline
PRI3 This site protects information about my credit card. \\ \hline
\rowcolor[HTML]{C0C0C0} 
\multicolumn{1}{|c|}{\cellcolor[HTML]{C0C0C0}Responsiveness} \\ \hline
RES1 It provides me with convenient options for returning items. \\ \hline
RES2 This site handles product returns well. \\ \hline
RES3 This site offers a meaningful guarantee. \\ \hline
RES4 It tells me what to do if my transaction is not processed. \\ \hline
RES5 It takes care of problems promptly. \\ \hline
\rowcolor[HTML]{C0C0C0} 
\multicolumn{1}{|c|}{\cellcolor[HTML]{C0C0C0}Compensation} \\ \hline
COM1 This site compensates me for problems it creates. \\ \hline
COM2 It compensates me when what I ordered doesn’t arrive on time. \\ \hline
COM3 It picks up items I want to return from my home or business. \\ \hline
\rowcolor[HTML]{C0C0C0} 
\multicolumn{1}{|c|}{\cellcolor[HTML]{C0C0C0}Contact} \\ \hline
CON1 This site provides a telephone number to reach the company. \\ \hline
CON2 This site has customer service representatives \textbackslash{}available online. \\ \hline
CON3 It offers the ability to speak to a live person if there is a problem. \\ \hline
\end{tabular}%
}
\end{table}

\chapter{Quinto Anexo}
\pagebreak
% Please add the following required packages to your document preamble:
% \usepackage{graphicx}
% \usepackage[table,xcdraw]{xcolor}
% If you use beamer only pass "xcolor=table" option, i.e. \documentclass[xcolor=table]{beamer}
\begin{table}[]
\caption{Instrumento de aferição de qualidade para serviços eletrônicos com foco na perspectiva do consumidor. Fonte: \cite{zhang2005}}
\label{table:zhang2005}
\resizebox{1.1\textwidth}{!}{%
\begin{tabular}{|l|}
\hline
\rowcolor[HTML]{9B9B9B} 
\multicolumn{1}{|c|}{\cellcolor[HTML]{9B9B9B}A Consumer Perspective of E-Service Quality(Zhang)} \\ \hline
\rowcolor[HTML]{C0C0C0} 
\multicolumn{1}{|c|}{\cellcolor[HTML]{C0C0C0}Website service quality} \\ \hline
It is easy to navigate in this site \\ \hline
The information about the products for your needs/interests is sufficient  for me to make a purchase decision \\ \hline
The site is visually appealing \\ \hline
The information about the products/services for your needs/interests is sufficient \\ \hline
I am happy with the vendor guaranteed policy \\ \hline
The vendor gives prompt service to customers \\ \hline
This site seems to be up to date \\ \hline
\rowcolor[HTML]{C0C0C0} 
\multicolumn{1}{|c|}{\cellcolor[HTML]{C0C0C0}Perceived Risk} \\ \hline
I worry about credit card information being stolen on internet \\ \hline
I worry about the product quality in internet \\ \hline
I worry about safe transaction online \\ \hline
I worry about how online merchants might use the personal information they obtain when I buy online \\ \hline
\rowcolor[HTML]{C0C0C0} 
\multicolumn{1}{|c|}{\cellcolor[HTML]{C0C0C0}Intention} \\ \hline
I intend to use E-service frequently \\ \hline
I intend to use E-service \\ \hline
In the future, I intend to use E-service whenever I need \\ \hline
\rowcolor[HTML]{C0C0C0} 
\multicolumn{1}{|c|}{\cellcolor[HTML]{C0C0C0}E-Service Convenience} \\ \hline
Using the internet makes it easier for me to shop \\ \hline
Online shopping is convenient \\ \hline
Shopping online saves time compared to going to a traditional store \\ \hline
\rowcolor[HTML]{C0C0C0} 
\multicolumn{1}{|c|}{\cellcolor[HTML]{C0C0C0}E-satisfaction} \\ \hline
I am satisfied with my previous online shopping experience \\ \hline
Online shopping is a pleasant experience \\ \hline
Overall I am satisfied with my E-service experience \\ \hline
\end{tabular}%
}
\end{table}

\end{anexosenv}
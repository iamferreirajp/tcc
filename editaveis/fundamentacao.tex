\chapter[Fundamentação teórica]{Fundamentação teórica}

\section{Governo Digital}
Ao iniciarmos esse estudo faz-se necessário a definição do que é o governo digital. Ter conhecimento sobre tal conceito é de extrema relevância, pois nos permite entender o contexto que origina os serviços digitizados e suas formas de avaliação, que é o foco deste trabalho. Além disso é perceptível a importância do assunto e o impacto que pode causar em uma sociedade e em seu governo.

Concordando com \cite{itac2016} para que um governo se mantenha competitivo no cenário mundial e que tenha engajamento dos cidadãos é de crítica importância sua característica digital. Na economia moderna há vantagens na construção de um governo realmente digital, como oportunidades de transformação, modernização e digitização que ajudam no crescimento e na escalabilidade de uma organização.

\subsection{Definição}

Ainda de acordo com \cite{itac2016} existe uma definição de um governo digital. Tal denotação se baseia em 4 características principais:
A facilidade de se trabalhar com o governo
A facilidade de se trabalhar no governo
As tecnologias são sempre atualizadas 
A informação governamental é digitizada e facilmente encontrável.

Portanto são áreas que devem nortear o investimento governamental para se conquistar um governo realmente digital.
 
O Comitê de Governança Pública em sua Recomendação para Estratégias de Governo Digital \cite{oecd2014} destaca as vantagens da adoção das novas tecnologias no governo. As vantagens extrapolam o valor somente para a organização, alcançando o cidadão, as empresas e até organizações não-governamentais. O ambiente digital oferece oportunidades para que todos participem de decisões de priorização política, no desenho de serviços públicos e permite que o governo possa antecipar as necessidades dos interessados em seus negócios.

A digitização do governo, entretanto, não é instalada por inteiro na estrutura governamental, de acordo com \cite{mckinsey2016} os esforços se iniciam na reconstrução das capacidades básicas com novas tecnologias. Com a experiência adquirida dentro da instituição é possível então entregar a experiência de ser digital para os cidadãos, empresas ou qualquer usuário dos serviços disponíveis.

\section{Avaliação de qualidade de serviços}
\subsection{Conceitos básicos}

Há uma crescente consciencia de que bens e serviços de alta qualidade podem dar a uma organização uma considerável vantagem competitiva. Boa qualidade reduz custos de retrabalaho, refugo e devoluções e, mais importante, boa qualidade gera consumidores satisfeitos. Alguns gerentes de produção acreditam que a longo prazo a qualidade é o mais importante fator singular que afeta o desempenho de uma organização em relação a seus concorrentes. \cite[p.~150]{slack2002}

\apud{slack2002}{mckinsey2016}

\subsubsection{Qualidade de serviço}
\subsubsection{Dimensões de avaliação da qualidade}

\subsection{SERVQUAL: Modelo de avaliação da qualidade de serviço}
\subsubsection{Concepção do modelo}
\subsubsection{Instrumento: Dimensões e questões}
\subsubsection{Problemas e vantagens do instrumento}

\subsection{SERVPERF: Modelo de avaliação derivado do SERVQUAL}
\subsubsection{}


\section{Serviços digitizados}
\subsection{Serviços digitizados}
\subsection{Características}

\section{Governo para o cidadão}
\subsection{Utilização do serviço para o cidadão}
\subsection{Desafio da avaliação}

\section{Avaliação de qualidade de serviços digitizados}
\subsection{Serviço digitizado X Serviço tradicional}
\subsection{Modelo preliminar}
\subsubsection{Descrição do modelo}
\subsubsection{Estatísticas relacionadas}